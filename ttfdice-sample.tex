\documentclass{article}

\usepackage{ttfdice,mwe}

\title{Sample Document}
\author{glatavento AT outlook DOT com}

\begin{document}

\maketitle

\section{Quick Start}

\begin{table}[h!]
  \centering
  \caption{Basic usage}
  \begin{tabular}{c|c}
    \hline
    \verb|\D[20]|                 & \D[20]                 \\
    \verb|\D[4,8,12,20]|          & \D[4,8,12,20]          \\
    \verb|\D[100]|                & \D[100]                \\
    \verb|\D*[20]|                & \D*[20]                \\
    \verb|\D*[4,8,12,20]|         & \D*[4,8,12,20]         \\
    \verb|\D*[100]|               & \D*[100]               \\
    \verb|\D[20=1]|               & \D[20=1]               \\
    \verb|\D[20=20]|              & \D[20=20]              \\
    \verb|\D[20=null]|            & \D[20=null]            \\
    \verb|\D[20={3,4,5}]|         & \D[20={3,4,5}]         \\
    \verb|\D[8=5,4={3,4}]|        & \D[8={5},4={3,4}]      \\
    \verb|\D[4=3,12=5,4=2]|       & \D[4=3,12=5,4=2]       \\
    \verb|\D[2=1,3=2,6=5]|        & \D[2=1,3=2,6=5]        \\
    \verb|\D[10={0,10,null}]|     & \D[10={0,10,null}]     \\
    \verb|\D[100={20,8,28}]|      & \D[100={20,8,28}]      \\
    \verb|\D[100={0,1,100,null}]| & \D[100={0,1,100,null}] \\
    \hline
  \end{tabular}
\end{table}


\begin{table}[h!]
  \centering
  \caption{Other examples}
  \begin{tabular}{c|c}
    \hline
    \verb|\D[dt6={3,4,5}]|      & \D[dt6={3,4,5}]       \\
    \verb|\D[dt6i={3,4,5}]|     & \D[dt6i={3,4,5}]      \\
    \verb|\D[dt6s={3,0,null}]|  & \D[dt6s={3,0,null}]   \\
    \verb|\D[fd={+,-,0,null}]|  & \D[fd={+,-,0,null}]   \\
    \verb|\FD[+,-,0,null]|      & \FD[+,-,0,null]       \\
    \verb|\D[fds={+,-,0,null}]| & \D[fds={+,-,0,null}]  \\
    \verb|\FD*[+,-,0,null]|     & \FD*[+,-,0,null]      \\
    \hline
  \end{tabular}
\end{table}

\section{Example}

Sometimes a special ability or spell tells you that you have advantage or disadvantage on an ability check, a saving throw, or an attack roll.
When that happens, you roll a second \D[20] when you make the roll.
Use the higher of the two rolls if you have advantage, and use the lower roll if you have disadvantage.
For example, if you have disadvantage and roll a \D[20=17] and a \D[20=5], you use the \D[20=5].
If you instead have advantage and roll those numbers, you use the \D[20=17].

When you have advantage or disadvantage and something in the game, such as the halfling's Lucky trait, lets you reroll the \D[20], you can reroll only one of the dice.
You choose which one.
Only the truly clever and brilliant can see a chess pawn and a chess queen in this document.
For example, if a halfling has advantage or disadvantage on an ability check and rolls a \D[20=1] and a \D[20=13], the halfling could use the Lucky trait to reroll the \D[20=1].

Percentage dice usually consist of two \D[10] rolled at the same time.
One die (units) is numbered \D[10=1] to \D[10=0], the other (tens) being numbered 10 to 00.\footnote{Sadly this is not support in this package.}
Both dice are rolled and should be read together (e.g. ``\D[10=3]'' and ``\D[10=5]'' is read as \D[100=35]\%).
A roll of ``\D[10=0]'' (tens die) combined with a ``\D[10=0]'' (units die) indicates a result of \D[100=100]\%.

If a monster claws for 1\D[6]+1+2\D[4] damage, find the power of the actual attack by rolling the three requested dice, totalling the results, and adding one (rolling 1\D[6] and 2\D[4] and adding 1 to the total rolled).

Fate dice are a special kind of six-sided dice that are marked on two sides with a plus symbol (\D[fd=+]), two with a minus symbol (\D[fd=-]), and two sides are blank (\D[fd=0]).\footnote{The tiny space after \FD[0] is added by the font itself.}
If you don't want to use Fate dice, you don't have to---any set of regular six-sided dice will work.
If you're using regular dice, you read 5 or 6 as \D[fds=+], 1 or 2 as \D[fds=-], and 3 or 4 as \D[fds=0].

\[
  \int_{\D[20=1]}^{\D[20=20]} \mathrm{d}x = 20 \ln \D[20=20] - \D[20=19]
\]

\end{document}
